% Options for packages loaded elsewhere
\PassOptionsToPackage{unicode}{hyperref}
\PassOptionsToPackage{hyphens}{url}
%
\documentclass[
]{book}
\usepackage{amsmath,amssymb}
\usepackage{iftex}
\ifPDFTeX
  \usepackage[T1]{fontenc}
  \usepackage[utf8]{inputenc}
  \usepackage{textcomp} % provide euro and other symbols
\else % if luatex or xetex
  \usepackage{unicode-math} % this also loads fontspec
  \defaultfontfeatures{Scale=MatchLowercase}
  \defaultfontfeatures[\rmfamily]{Ligatures=TeX,Scale=1}
\fi
\usepackage{lmodern}
\ifPDFTeX\else
  % xetex/luatex font selection
\fi
% Use upquote if available, for straight quotes in verbatim environments
\IfFileExists{upquote.sty}{\usepackage{upquote}}{}
\IfFileExists{microtype.sty}{% use microtype if available
  \usepackage[]{microtype}
  \UseMicrotypeSet[protrusion]{basicmath} % disable protrusion for tt fonts
}{}
\makeatletter
\@ifundefined{KOMAClassName}{% if non-KOMA class
  \IfFileExists{parskip.sty}{%
    \usepackage{parskip}
  }{% else
    \setlength{\parindent}{0pt}
    \setlength{\parskip}{6pt plus 2pt minus 1pt}}
}{% if KOMA class
  \KOMAoptions{parskip=half}}
\makeatother
\usepackage{xcolor}
\usepackage{longtable,booktabs,array}
\usepackage{calc} % for calculating minipage widths
% Correct order of tables after \paragraph or \subparagraph
\usepackage{etoolbox}
\makeatletter
\patchcmd\longtable{\par}{\if@noskipsec\mbox{}\fi\par}{}{}
\makeatother
% Allow footnotes in longtable head/foot
\IfFileExists{footnotehyper.sty}{\usepackage{footnotehyper}}{\usepackage{footnote}}
\makesavenoteenv{longtable}
\usepackage{graphicx}
\makeatletter
\def\maxwidth{\ifdim\Gin@nat@width>\linewidth\linewidth\else\Gin@nat@width\fi}
\def\maxheight{\ifdim\Gin@nat@height>\textheight\textheight\else\Gin@nat@height\fi}
\makeatother
% Scale images if necessary, so that they will not overflow the page
% margins by default, and it is still possible to overwrite the defaults
% using explicit options in \includegraphics[width, height, ...]{}
\setkeys{Gin}{width=\maxwidth,height=\maxheight,keepaspectratio}
% Set default figure placement to htbp
\makeatletter
\def\fps@figure{htbp}
\makeatother
\setlength{\emergencystretch}{3em} % prevent overfull lines
\providecommand{\tightlist}{%
  \setlength{\itemsep}{0pt}\setlength{\parskip}{0pt}}
\setcounter{secnumdepth}{5}
\usepackage{booktabs}
\usepackage{amsthm}
\makeatletter
\def\thm@space@setup{%
  \thm@preskip=8pt plus 2pt minus 4pt
  \thm@postskip=\thm@preskip
}
\makeatother
\ifLuaTeX
  \usepackage{selnolig}  % disable illegal ligatures
\fi
\usepackage[]{natbib}
\bibliographystyle{apalike}
\IfFileExists{bookmark.sty}{\usepackage{bookmark}}{\usepackage{hyperref}}
\IfFileExists{xurl.sty}{\usepackage{xurl}}{} % add URL line breaks if available
\urlstyle{same}
\hypersetup{
  pdftitle={Machine Learning Guidelines for Natural Resource Management Practitioners},
  pdfauthor={Shih-Ni Prim and Natalie Nelson},
  hidelinks,
  pdfcreator={LaTeX via pandoc}}

\title{Machine Learning Guidelines for Natural Resource Management Practitioners}
\author{Shih-Ni Prim and Natalie Nelson}
\date{2024-01-17}

\begin{document}
\maketitle

{
\setcounter{tocdepth}{1}
\tableofcontents
}
\hypertarget{motivation}{%
\chapter{Motivation}\label{motivation}}

As machine learning (ML) has become a powerful tool, it is noted by some that ML has not been widely used in environmental studies. This booklet is meant to provide a concise guide for natural resource management practitioners. This book serves as a staring point rather than a comprehensive resource, so that practitioners can have a basic understanding of how ML works and how to utilize it to analyze data and answer research questions. When appropriate, we provide case studies and R code as well as other online resources to help the readers on the journey of gaining one powerful tool that seems to be omnipresent in the research world.

\hypertarget{intro}{%
\chapter{Introduction}\label{intro}}

What is machine learning?

\hypertarget{supervised-learning}{%
\section{Supervised Learning}\label{supervised-learning}}

\hypertarget{unsupervised-learning}{%
\section{Unsupervised Learning}\label{unsupervised-learning}}

\hypertarget{data}{%
\chapter{Data}\label{data}}

\hypertarget{what-to-do-with-data}{%
\section{What to do with data?}\label{what-to-do-with-data}}

One could argue that data is the single most important ingredient when it comes to machine learning models or any type of analysis. As one might say, junk in, junk out.

\hypertarget{data-requirement}{%
\section{Data Requirement}\label{data-requirement}}

\hypertarget{evaluation}{%
\chapter{Evaluation}\label{evaluation}}

\hypertarget{continuous-responses}{%
\section{Continuous Responses}\label{continuous-responses}}

\hypertarget{discrete-responses}{%
\section{Discrete Responses}\label{discrete-responses}}

\hypertarget{cross-validation}{%
\section{Cross Validation}\label{cross-validation}}

\hypertarget{machine-learning-methods}{%
\chapter{Machine Learning Methods}\label{machine-learning-methods}}

Here we provide a list of commonly used machine learning methods and some brief discussion.

\hypertarget{random-forest}{%
\section{Random Forest}\label{random-forest}}

\hypertarget{presentation}{%
\chapter{Presentation}\label{presentation}}

It is also important to present the results in a way that aids rather than impede communication.

\hypertarget{table}{%
\section{Table}\label{table}}

\hypertarget{figure}{%
\section{Figure}\label{figure}}

\hypertarget{ethical-considerations}{%
\chapter{Ethical Considerations}\label{ethical-considerations}}

\hypertarget{reproducibility}{%
\section{Reproducibility}\label{reproducibility}}

\hypertarget{appendix}{%
\chapter{Appendix}\label{appendix}}

\hypertarget{dos-and-donts}{%
\section{Do's and Don'ts}\label{dos-and-donts}}

  \bibliography{book.bib,packages.bib}

\end{document}
